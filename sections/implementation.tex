\clearpage
\section{Prototype Implementation}
\label{sec:implementation}

\subsection{Frontend}
The following components have been implemented for the front end:

\begin{itemize}
	\item Login: this is the first view that the user are presented with. It’s main purpose is to
authenticate users, and help them access the application. It contains input fields for
username and password. When the user presses “Log in”, he is taken to the main-page of the
application. If the user is not registered, he can easily do so by pressing the “Register”-
button.
	\item Register: here the user is presented a schema, where he can register and that way access the
application. He needs to add an email address, a username and a password. Once the
information is submitted, the user is sent back to the login page.
	\item Main-page: This component contains buttons that lets the user easily navigate to all parts of
the application quickly.
	\item Find-poll: Here the user can search for a poll via its id, or by a poll-name. All the polls that
matches the input will then be displayed with the option to vote on it.
	\item Create-poll: Here the user can create a new poll. He can decide the questions that should be
displayed, when the poll is active, if it is private and invite users to participate in it. Once
created, a poll-id is given to the user. This can for example be given to other users and used
to search for the poll.
	\item Vote: Here the user gets to submit votes to active polls.
\end{itemize}

A authentication service, a poll service and a voting service has also been created to handle logic that
can be applied to multiple components. The authentication service handles operations such as
logging the user in to the application and searching for users. The poll service handles logic depicting
the polls such as finding polls, creating polls and changing polls. The Voting service handles the logic
connected to the voting.

The following subsection describes more detailed how the logic behind the authentification process has been implemented, 
and displays how the frontend is connected to the REST API. 

\subsubsection{Authentication Process}
A crucial part of the frontend implementation is the user authentication process. 
This is managed through the \texttt{login} method in the \texttt{auth.service} file and the \texttt{onSubmit} method in the \texttt{LoginComponent}.

The \texttt{login} method is defined as:
\begin{verbatim}
login(user: { username: string, password: string }) {
  return this.http.post<string>(`${this.apiUrl}/login`, user, {
    headers: new HttpHeaders({ 'Content-Type': 'application/json' }),
    responseType: 'text' as 'json'
  }).pipe(
    tap(response => {
      localStorage.setItem('authToken', response);
      localStorage.setItem('username', user.username);
      console.log('Server response:', response);
    })
  );
}
\end{verbatim}

This method handles user login credentials, sending them to the backend for verification and storing the received token and 
username in the browser's local storage for session management. The token is stored in the local storage to maintain the user sessions. 

The \texttt{LoginComponent} uses this service in its \texttt{onSubmit} method:
\begin{verbatim}
onSubmit() {
  if (this.loginForm.valid) {
    this.authService.login(this.loginForm.value)
      .subscribe(
        (token: string) => {
          if (token) {
            localStorage.setItem('authToken', token);
            this.router.navigate(['/main-page']);
          } else {
            alert('Incorrect username or password.');
          }
        },
        (error: any) => {
          if (error.status === 401) {
            alert('Incorrect username or password.');
          } else {
            alert('An error occurred during login.');
          }
        }
      );
  }
}
\end{verbatim}

This function makes sure that only users with valid credentials are able to access the main page of the application, while also handling 
different error scenarios. If a user gives an incorrect credential or any other login error occurs, an error message is displayed. 


There are also a few features that we modelled into our application in the first phase of this project,
but due to the time constraints are yet to be implemented. These include Poll management and User
management. Currently the application allows users to create and participate in polls. However, the
management of the polls is not yet functional. The same goes for the management of the user
settings. The buttons has been created in the main-page component, but they are not directing the
users to new views yet.